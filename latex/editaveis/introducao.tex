\chapter*[Introdução]{Introdução}
\addcontentsline{toc}{chapter}{Introdução}

Bem-vinda(o) ao seu trabalho de TCC. Leia atentamente as instruções presentes nesse documento.

O seu trabalho deve iniciar com alguns parágrafos sobe e a
Contextualização. Fale sobre onde seu trabalho está inserido e prepare o leitor para os problemas que serão resolvidos no seu trabalho. NÃO crie uma seção chamada 1.1 Contextualização. Apenas comece o texto.

\section{Metodologia}
Metodologia: seção OPCIONAL. Em trabalho científicos, os métodos utilizados podem levar a (esperados) resultados diferentes. Tradicionalmente adiciona-se uma seção de metodologia para descrever métodos, instrumentos e formas de medição que iram guiar o trabalho.

Na computação, esses métodos podem ser incluídos rapidamente em uma seção de resultados, em uma seção de ambientes computacional  com a CPU, RAM e formas de medição. Por isso, é muito comum remover a metodologia da introdução. Mantenha-a apenas se tiver informações relevantes.

\section{Justificativa}
Seção incluída tradicionalmente em trabalhos científicos, opcional em trabalhos de computação.

\section{Objetivos}
Seção OBRIGATÓRIA. Explique ao leitor, qual objetivo você quer resolver em seu trabalho.

\subsection{Objetivos específicos}
\begin{itemize}
\item O objetivo geral deve ser divido em objetivos menores.
\item Esses objetivos menores devem ser apresentados em itens.
\item Lembre-se! Ao final do seu trabalho (TCC2) é importante concluir que todos objetivos foram atingidos. Escolha-os atentamente.
\end{itemize}

\section{Estrutura do Documento}
Este documento está estruturado da seguinte forma: no Capítulo \ref{ch:fund} é apresentada a fundamentação teórica do seu trabalho. No Capítulo \ref{ch:proposta} é apresentada a proposta do trabalho. No Capítulo \ref{ch:resultados} são apresentados os resultados e análise dos resultados. Finalmente, no Capítulo \ref{ch:conclusao} apresentamos a conclusão e trabalhos futuros.

%Use % para fazer um comentário no seu latex e TODO
%TODO: REMOVER o verbatim abaixo no seu trabalho.
\begin{verbatim}
Instruções: para produzir a estrutura acima é necessário usar o comando \ref{ch:umnome} ou \ref{ch:outronome} no parágrafo e adicionar o \label{ch:umnome} e o \label{ch:outronome} em cada um dos capítulos do seu texto.
\end{verbatim}
