\chapter{Aspectos Gerais}

Estas instruções apresentam um conjunto mínimo de exigências necessárias a 
uniformidade de apresentação do relatório de Trabalho de Conclusão de Curso 
da FCTE. Estilo, concisão e clareza ficam inteiramente sob a 
responsabilidade do(s) aluno(s) autor(es) do relatório.

As disciplinas de Trabalho de Conclusão de Curso (TCC) 01 e Trabalho de 
Conclusão de Curso (TCC) 02 se desenvolvem de acordo com Regulamento 
próprio aprovado pelo Colegiado da FGA. Os alunos matriculados nessas 
disciplinas devem estar plenamente cientes de tal Regulamento. 

\section{Composição e estrutura do trabalho}

A formatação do trabalho como um todo considera três elementos principais: 
(1) pré-textuais, (2) textuais e (3) pós-textuais. Cada um destes, pode se 
subdividir em outros elementos formando a estrutura global do trabalho, 
conforme abaixo (as entradas itálico são \textit{opcionais}; em itálico e
negrito são \textbf{\textit{essenciais}}):

\begin{description}
	\item [Pré-textuais] \

	\begin{itemize}
		\item Capa
		\item Folha de rosto
		\item \textit{Dedicatória}
		\item \textit{Agradecimentos}
		\item \textit{Epígrafe}
		\item Resumo
		\item Abstract
		\item Lista de figuras
		\item Lista de tabelas
		\item Lista de símbolos e
		\item Sumário
	\end{itemize}

	\item [Textuais] \

	\begin{itemize}
		\item \textbf{\textit{Introdução}}
		\item \textbf{\textit{Fundamentação Teórica}}
		\item \textbf{\textit{Revisão da Literatura}}
		\item \textbf{\textit{Proposta de Trabalho}}
		\item \textbf{\textit{Resultados Obtidos}}
		\item \textbf{\textit{Conclusão e Trabalhos Futuros}}
	\end{itemize}

	\item [Pós-Textuais] \
	
	\begin{itemize}
		\item Referências bibliográficas
		\item \textit{Bibliografia}
		\item Anexos
		\item Contracapa
	\end{itemize}
\end{description}

No modelo \LaTeX, os arquivos correspondentes a estas estruturas que devem
ser editados manualmente estão na pasta \textbf{editáveis}. Os arquivos
da pasta \textbf{fixos} tratam os elementos que não necessitam de 
edição direta, e devem ser deixados como estão na grande maioria dos casos.

\section{Quebra de Capítulos e Aproveitamento de Páginas}

Cada seção ou capítulo deverá começar numa nova pagina (recomenda-se que 
para texto muito longos o autor divida seu documento em mais de um arquivo 
eletrônico). 

Caso a última pagina de um capitulo tenha apenas um número reduzido de 
linhas (digamos 2 ou 3), verificar a possibilidade de modificar o texto 
(sem prejuízo do conteúdo e obedecendo as normas aqui colocadas) para 
evitar a ocorrência de uma página pouco aproveitada.

Ainda com respeito ao preenchimento das páginas, este deve ser otimizado, 
evitando-se espaços vazios desnecessários. 

Caso as dimensões de uma figura ou tabela impeçam que a mesma seja 
posicionada ao final de uma página, o deslocamento para a página seguinte 
não deve acarretar um vazio na pagina anterior. Para evitar tal ocorrência, 
deve-se reposicionar os blocos de texto para o preenchimento de vazios. 

Tabelas e figuras devem, sempre que possível, utilizar o espaço disponível 
da página evitando-se a \lq\lq quebra\rq\rq\ da figura ou tabela. 
